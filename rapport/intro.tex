
\chapter{Introduction}

\section{Présentation du challenge}

PriceMinister est une entreprise et plateforme d'achat-vente en ligne jouant le rôle 
d'intermédiaire entre acheteurs et vendeurs comprenant 17 millions d'utilisateurs inscrits
en 2016. Avec environ 50 000 transactions par jour de produits classés an 14 rubriques 
(Informatique, électroménager, loisirs, beauté, ...) c'est l'une des plateformes de commerce
en ligne majeure en France. 

Ce challenge vise à prédire si une transaction passée en ligne via le site PriceMinister
a des chances d'aboutir à une réclamation de la part de l'acheteur, par exemple un colis 
non reçu ou endommagé, et le cas échéant prédire le type de réclamation. Améliorer la 
capacité de prédiction des réclamations est un des enjeux majeurs de l'e-commerce car il 
permet non seulement d'améliorer l'expérience utilisateur, mais aussi d'augmenter le chiffre 
d'affaires en anticipant d'éventuels coûts en se focalisant sur les transactions à risques.

\section{Présentation du jeu de données}

Il s'agit donc d'un problème de classification multi-classe dont les classes à prédire sont :

\begin{itemize}
\item '-' : pas de réclamation
\item 'WITHDRAWAL' : retrait de la commande
\item 'SELLER_CANCEL_POSTERIORI' : le vendeur annule la commande
\item 'NOT_RECEIVED' : colis non reçu
\item ‘DIFFERENT' : produit différent de l'annonce
\item 'DAMAGED' : produit endommagé
\item 'FAKE' : le produit est un faux (arnaque)
\item 'UNDEFINED' : réclamation d'un autre type
\end{itemize}

Le jeu de données comprend des données hétérogènes telles que le moyen de paiement, les
départements ou pays du vendeur et de l'acheteur, le type de produit acheté :

\begin{itemize}
\item ID: identifiant de la commande
\item SHIPPING_MODE: méthode de transport
\item SHIPPING_PRICE: coût du transport, si existant
\item WARRANTIES_FLG: True si une garantie a été prise par l'acheteur
\item WARRANTIES_PRICE: prix de la garantie si existante
\item CARD_PAYEMENT: indicatrice du moyen de paiement par carte bancaire
\item COUPON_PAYEMENT: indicatrice du moyen de paiement avec un coupon discount 
\item RSP_PAYEMENT: indicatrice du moyen de paiement avec des Rakuten Super Points
\item WALLET_PAYMENT: indicatrice du moyen de paiement avec PriceMinister-Rakuten wallet
\item PRICECLUB_STATUS: status de l'acheteur
\item REGISTRATION_DATE: année d'enregistrement de l'acheteur
\item PURCHASE_COUNT: nombre d'achats précédemment réalisés par l'acheteur
\item BUYER_BIRTHDAY_DATE: année de naissance de l'acheteur
\item BUYER_DEPARTMENT: département de l'acheteur
\item BUYING_DATE: année et mois de l'achat
\item SELLER_SCORE_COUNT: nombre de ventes précédemment réalisées par le vendeur
\item SELLER_SCORE_AVERAGE: score du vendeur sur le site PriceMinister-Rakuten
\item SELLER_COUNTRY: pays du vendeur
\item SELLER_DEPARTMENT: département français du vendeur (-1 si vendeur à l'étranger)
\item PRODUCT_TYPE: type de produit commandé
\item PRODUCT_FAMILY: famille du produit commandé
\item ITEM_PRICE: prix du produit acheté

\end{itemize}

Plusieurs problématiques sont soulevées par ce jeu de données. Dans un premier temps, 
il est essentiel de mettre en relief l'ancienneté et la fiabilité à la fois du vendeur et 
de l'acheteur en croisant les données en rapport avec les statuts, le nombre d'années 
d'inscription sur le site et le nombre de ventes/commandes réalisées. De plus, les 
nombreuses données géographiques devront être exploitées pour mettre an avant la distance
parcouru par la commande. Le type et la famille de produit peuvent aussi s'avérer déterminant
dans la prédiction de réclamations étant données que certains produits doivent être 
plus sujets à certains types de plaintes telles que la casse. De la même manière certains
types de transports peuvent être plus sujets à la casse ou au retard de livraison. 
Enfin, le mode de paiement et l'achat d'une garantie peuvent aussi d'une manière moins
évidente influencer les réclamations.



