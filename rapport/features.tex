
\chapter{Feature Engineering}

Cette section présente les principales transformations du jeu de données mises en oeuvre.
Cela comprend en particulier la transformation des variables catégorielles en
indicatrices ou dummy variables (SHIPPING_MODE, PRODUCT_FAMILY,...), le mapping des variables 
ordinales en variables d'entiers (SHIPPING_PRICE, PURCHASE_COUNT, ...),
compléter les valeurs manquantes et enfin créer de nouvelles variables.

\textbf{Variables transformées :}

\begin{itemize}
\item REGISTRATION_DATE est transformée en BUYER_SENIORITY indiquant le nombre d'années
d'ancienneté de l'utilisateur sur le site plutôt qu'une année 
\item BUYER_BIRTHDAY_DATE est remplacée par BUYER_AGE pour la même raison
\end{itemize}

\textbf{Variables créées :}

\begin{itemize}
\item UNEXPERIENCED_BUYER pour les individus qui ont effectué moins de 5 achats sur le site
\item VIP_SELLER pour les vendeurs ayant une note de 49/50
\item BUYER_REGION et SELLER_REGION représentant les régions des vendeurs et acheteurs.
Cette variables vient renforcer les variables BUYER_DEPARTMENT et SELLER_DEPARTMENT.
\item BUYER_IS_ABROAD et SELLER_IS_ABROAD indiquant que le vendeur/acheteur est à l'étranger.
\item SAME_REGION_BUYER_SELLER et SAME_DEPARTMENT_BUYER_SELLER pour indiquer si l'acheteur
et le vendeur d'une commande sont issus da la même région ou du même département.
\item SELLER_BUYER_REGION_DISTANCE distance en kms entre la ville principale de la région
du vendeur et la ville principale de la région de l'acheteur. Cette variable, obtenue en 
réalisant une matrice de distance relatives entre chaque région permet d'apporter une
information supplémentaire quant à la distance approximativement parcourue lors de la 
livraison lorsque le vendeur et l'acheteur sont en France. 
\item SELLER_COUNTRY_DISTANCE indiquant la distance relative entre le pays du vendeur et la France. 
Cette distance est échelonnée suivant une échelle de 0 à 4, 0 étant pour FRANCE METROPOLITAINE et 4 pour les 
pays les plus éloignées tels que JAPON, AUSTRALIE. Cette variable vient donc prolonger la variable
SELLER_BUYER_REGION_DISTANCE lorsque le vendeur est à l'étranger.
\end{itemize}

\textbf{Remplacement des valeurs manquantes :}

\begin{itemize}
\item  BUYER_AGE : méthode ffill
\item WARRANTIES_PRICE : les valeurs manquantes correspondent au cas où l'utilisateur n'a 
pas pris de garantie (même quantité de valeurs manquantes qu'il n'y a de False dans la 
variable WARRANTIES_FLG) et sont donc mises à 0.
\item  SHIPPING_PRICE : nous avons considéré que les valeurs nulles correspondent au cas
où les frais de livraisons sont gratuits et donc mis ces valeurs à 0
\end{itemize}

\textbf{Cas particulier :}

PRICECLUB_STATUS peut être vu comme une variable catégorielle nominale et être transformée
en variables binaires. Cependant, il y a une hiérarchie entre les différents 
status : UNSUBSCRIBED < REGULAR < PLATINUM < SILVER < GOLD. Nous avons donc préféré transformer
cette variable en une variables d'entiers (0 pour UNSUBSCRIBED jusqu'à 4 pour GOLD).




