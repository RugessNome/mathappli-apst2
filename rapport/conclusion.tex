
\chapter{Conclusion}

Ce projet nous a permis de mettre en pratique nos acquis en machine learning et en data 
science en nous confrontons a un problème issu de l'industrie. Nous avons notamment réaliser
au cours de cette étude toute l'importance du feature engineering. L'ajout d'une seule variable
peut s'avérer déterminante, comme cela a été le cas avec les derniers features de comparaison
des distances entre acheteurs et vendeurs. Cette étape est aussi celle qui demande le plus
de temps et d'effort et ne doit pas être bâclée en se contentant de
transformer les variables catégorielles du jeu de donnée en variables binaires. De plus,
nous avons pu nous rendre compte que l'intuition et l'expérimentation est aussi 
essentielle en machine learning. Beaucoup d'idées peuvent sembler bonnes et suivre une
certaine logique mais il faut avant tout tester ces idées pour voir émerger un modèle
efficace. C'est dans cette démarche que nous avons pu constater l'efficacité de passer
sur un modèle de prédiction en deux temps.

Pour améliorer notre modèle nous aurions pu pousser plus loin l'étude des distance entre
vendeurs en acheteurs en calculant les distance entre départements plutôt qu'entre région.
Nous aurions pu aussi tenter de mettre en place une méthode plus avancée telle que le 
stacking qui est souvent utilisée pour gagner les challenge de data science. Cependant,
la deuxième place que nous occupons à l'heure actuelle est très encourageante.
